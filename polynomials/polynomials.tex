\documentclass{article}
\usepackage{amsmath}
\usepackage{listings}
\usepackage{xcolor}

\title{Polynomials}
\author{}
\date{}

\begin{document}

\maketitle

\section*{Polynomials}

A polynomial is an expression composed of algebraic terms, especially the sum of several terms that contain different powers of the same variable(s).

\subsubsection*{Example}
The following is a polynomial:
$$ 7x^4 + 4x^2 + 5 $$

\subsection*{Parts of a Polynomial}
\begin{itemize}
  \item \textbf{Term:} A single part of a polynomial. Also known as a monomial. In the above example, there are three non-zero terms; \( 7x^4 \), \( 4x^2 \) and \( 5 \).
  \item \textbf{Degree:} The exponent of the term with the highest power. The degree of a monomial is its own exponent. The degree of the polynomial above is 4.
  \item \textbf{Coefficient:} The number in front of the variable. In the above example, the coefficients are 7, 4 and 5 (note that the degree of the constant term is 0, and since \(x^0\) is 0).
\end{itemize}

We can present the same polynomial from above without modifying its value by adding terms such that there is a monomial with a unique degree ranging from the degree of the polynomial to zero using the zero product and zero exponent properties:
$$ 7x^4 + 0x^3 + 4x^2 + 0x^1 + 5x^0 $$

\subsection*{Representing Polynomials using Python Lists}

A polynomial can be represented using a Python list. The value of the item in the list corresponds with the coefficient of the term, and the index of the item in the list corresponds with the degree of the term. Since Python list indices start at 0, the degree of a term represented by an item in the list is the length of the list minus the index of the item minus 1.

\subsubsection*{Example}
Consider the polynomial:
$$ 7x^4 + 0x^3 + 4x^2 + 0x^1 + 5x^0 $$
This can be represented as a list in Python:
$$ [7, 0, 4, 0, 5] $$

\begin{itemize}
    \item The first item in the list, 7, represents the coefficient 7 in the first term of the polynomial (\( 7x^4 \)). The degree of the term is 4, which is the length of the list (5) minus the index of the item (0) minus 1 (\( 5 - 0 - 1 = 4 \)).
    \item The second item in the list, 0, represents the coefficient 0 in the second term of the polynomial (\( 0x^3 \)). The degree of the term is 3, which is the length of the list (5) minus the index of the item (1) minus 1 (\( 5 - 1 - 1 = 3 \)).
    \item The third item in the list, 4, represents the coefficient 4 in the third term of the polynomial (\( 4x^2 \)). The degree of the term is 2, which is the length of the list (5) minus the index of the item (2) minus 1 (\( 5 - 2 - 1 = 2 \)).
    \item The fourth item in the list, 0, represents the coefficient 0 in the fourth term of the polynomial (\( 0x^1 \)). The degree of the term is 1, which is the length of the list (5) minus the index of the item (3) minus 1 (\( 5 - 3 - 1 = 1 \)).
    \item The fifth item in the list, 5, represents the coefficient 5 in the fifth term of the polynomial (\( 5x^0 \)). The degree of the term is 0, which is the length of the list (5) minus the index of the item (4) minus 1 (\( 5 - 4 - 1 = 0 \)).
\end{itemize}

\section*{Multiplying Polynomials}

We can expand the product of two polynomials by applying the distributive law. The final result is the sum of all possible products obtained by multiplying each term from the first polynomial by each term from the second polynomial. The general multiplication rule is:
$$ (a + b)(c + d) = ac + ad + bc + bd $$ 

\subsection*{Example}
Consider the product of two polynomials:
$$ (3x^1+2x^0)(3x^4 + 0x^3 + 4x^2 + 0x^1 + 5x^0) $$


\end{document}
