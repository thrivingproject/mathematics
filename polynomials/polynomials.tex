\documentclass{article}
\usepackage{amsmath}
\usepackage{listings}
\usepackage{xcolor}

\title{Polynomials}
\author{}
\date{}

\begin{document}

\maketitle

\section*{Polynomials}

A polynomial is an expression composed of algebraic terms, especially the sum of several terms that contain different powers of the same variable(s).

\subsection*{Example}
Consider the polynomial:
$$ 3x^4 + 4x^2 + 5 $$
All powers can be added using the zero product and zero exponent properties:
$$ 3x^4 + 0x^3 + 4x^2 + 0x^1 + 5x^0 $$

\subsection*{Parts of a Polynomial}
\begin{itemize}
  \item \textbf{Degree:} The highest exponent of a variable. In the above example, the degree is 7.
  \item \textbf{Term:} A single part of a polynomial. In the above example, there are three non-zero terms; \( 3x^7 \), \( 4x^2 \) and \( 5 \).
  \item \textbf{Coefficient:} The number in front of the variable. In the above example, the coefficients are 3, 4 and 5 (note that the degree of the constant term is 0, and since \(x^0\) is 0).
\end{itemize}

\subsection*{Representing Polynomials using Python Lists}

A polynomial can be represented using a Python list. Each item in the list corresponds with a term in the polynomial.

\subsection*{Example 1}
Consider the polynomial:
$$ 3x + 2 $$
This can be represented as a list in Python:
$$[3, 2] $$

\begin{itemize}
  \item The first item in the list, 3, represents the coefficient 3 in the first term of the polynomial (\( 3x \)).
  \item The second item in the list, 2, represents the coefficient 2 in the second term \( 2 \) of the polynomial (2).
\end{itemize}

\subsection*{Example 2}
Consider the polynomial:
$$ 3x^4 + 0x^3 + 4x^2 + 0x^1 + 5x^0 $$
This can be represented as a list in Python:
$$ [3, 0, 4, 0, 5] $$

\section*{Multiplying Polynomials}

Now that we know how to represent polynomials using Python lists, we can multiply two polynomials.
Consider two polynomials:
\begin{itemize}
  \item \( 3x + 2 \) represented as \([3, 2]\), and
  \item \( 3x^4 + 4x^2 + 5 \) represented as \([3, 0, 4, 0, 5]\)
\end{itemize}
The multiplication of these polynomials involves multiplying each term of the first polynomial with each term of the second and summing the results according to the powers of \( x \).

\end{document}
